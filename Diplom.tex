\documentclass[a4paper, 14pt]{article}
\usepackage[english, russian]{babel}
\usepackage[utf8x]{inputenc}
\usepackage{fullpage}
\usepackage{indentfirst} % Первый абзац в разделе тоже с красной строки
\usepackage{cmap} % для кодировки шрифтов в pdf (чтобы не было крокозябры при копировании из pdf )
\usepackage{graphicx} % для вставки картинок
\graphicspath{{img/}} % Путь к папке с картинками
\sloppy % Включение переноса слов в тексте

\title{Фреймворк для конечно-разностного моделирования диффузионных задач на гибридных вычислительных кластерах}
\author{Фролов Даниил Александрович}

% Для красивой вставки исходного программного кода 
\usepackage{listings}
\usepackage{color}

\definecolor{mygreen}{rgb}{0,0.6,0}
\definecolor{mygray}{rgb}{0.5,0.5,0.5}
\definecolor{mymauve}{rgb}{0.58,0,0.82}

% Параметры раскрски исходного кода программы 
\lstset{ %
	language = Java,
	extendedchars=\true, %Чтобы русские буквы в комментариях были
	%inputencoding=cp1251,
	%commentstyle=\itshape,
	%stringstyle=\bf,
  backgroundcolor=\color{white},   % choose the background color
  basicstyle=\footnotesize,        % size of fonts used for the code
  %basicstyle=\ttfamily\fontsize{11pt}{11pt}\selectfont,
  breaklines=true,                 % automatic line breaking only at whitespace
  captionpos=b,                    % sets the caption-position to bottom
  commentstyle=\color{mygreen},    % comment style
  escapeinside={\%*}{*)},          % if you want to add LaTeX within your code
  keywordstyle=\color{blue},       % keyword style
  stringstyle=\color{mymauve},     % string literal style
}

\usepackage{hyperref} % Для добавления ссылок в тесте

% Настройка цветов для ссылок
\hypersetup{
colorlinks = true,
linkcolor = black,
pagecolor = black,
urlcolor = blue, 
citecolor = black
}

% Математика
\usepackage{amssymb} % For use "mathbb" function
\usepackage{amsmath}
\usepackage{amsthm}
\usepackage{mathrsfs}
\newcommand{\La}{\mathscr{L}} % Функция Лагранжа
\newcommand{\ls}{{ℓ}} % Красивая l, чтобы легче было отличить от i, 1 b других палок
\providecommand{\norm}[1]{\lVert#1\rVert} % Норма вектора : ||w||
\newcommand{\dpt}[1]{\left\langle#1\right\rangle} % dot product using brackets
\newcommand{\brackets}[1]{\left(#1\right)} % Обернуть скобками автоматического размера
%\newcommand{\dpts}[2]{#2 \langle#1 #2 \rangle} % dot product using brackets with manual size
\newcommand{\R}{\mathbb{R}} % beautiful R for R^n labels
\newcommand{\il}{i = 1, \ldots, \ls} % writes i = 1, ..., l
\newcommand{\ili}{\quad i = 1, \ldots, \ls} % writes i = 1, ..., l with indent in begin
\newcommand{\sumil}{\sum_{i=1}^{\ls}} % Сумма по i, которая изменяется от 1 до l
\newcommand{\minl}{\min\limits} % min with limits under "min" label
\newcommand{\maxl}{\max\limits} % max with limits under "max" label

% Окружение для теорем, определений и т.д.
\theoremstyle{definition}
\newtheorem{definition}{Определение}
\newtheorem{theorem}{Теорема}
\newtheorem{example}{Пример}

% Поправить стиль отрисовки формул (особенно актуально для сумм https://ru.sharelatex.com/learn/Display_style_in_math_mode)
\everymath{\displaystyle}

%\bibliographystyle{unsrt} % упорядочить список использованной литературы по порядку упоминания их в тексте
\bibliographystyle{utf8gost705u}
%\bibliographystyle{utf8gost71u}

\usepackage[labelfont=bf, labelsep=space]{caption} % Делаем надписи "Рис.1" под рисунками жирными и без двоеточия.
\usepackage[top=20mm, bottom=20mm, left=30mm, right=20mm, nohead, nofoot]{geometry} % Размер полей у старницы
\setlength{\parindent}{1.25cm} % Размер интервала для абзацев 
\usepackage{setspace}
\singlespacing % одинарный интервал

\usepackage{caption} % подписи к рисункам в русской типографской традиции
\DeclareCaptionFormat{GOSTtable}{#2#1\\#3}
\DeclareCaptionLabelSeparator{fill}{\hfill}
\DeclareCaptionLabelFormat{fullparents}{\bothIfFirst{#1}{~}#2}
\captionsetup[table]{
     format=GOSTtable,
     %font={footnotesize},
     labelformat=fullparents,
     labelsep=fill,
     labelfont=normal,
     textfont=bf,
     justification=centering,
     singlelinecheck=false
     }

\begin{document}
\fontsize{14}{16pt}\selectfont
\section*{Введение}

\par Развитие современного общества зачастую ставит перед наукой цели, решение которых требует решения самых разнообразных систем дифференциальных уравнений. В том числе и систем диффузионных уравнений. К таким задачам можно отнести уравнение теплопроводности !!! (примеры других задач). Каждую из них можно решать с помощью численных методов, например, методом Эйлера, многошаговыми методами Рунге-Кутты или методами Дормана-Принца. Подобные вычисления удобно автоматизировать, чтобы в дальнейшем иметь возможность быстро производить расчеты.

\newpage
\section{Постановка задачи}
\par Требовалось создать программный комплекс, с помощью которого можно было бы автоматизировать моделирования конечно--разностых диффузионных задач. Для получения результата может быть применен один из следующих методов численного решения дифферинциальных уравнений, а именно: метод Эйлера, четырехстадийныз метод Рунге-Кутты, семистадийный метод Дормана-принца.
\par Разрабатываемый програмный комплекс должен поддерживать работу с уравнениями и системами уравнений, который были бы распределены в одномерных, двумерных и трехмерных областях. Каждая из таких областей может быть представлена неким набором блоков. Для одномерного случая этими блоками являются отрезки, в случае плоскости - прямоугольники, параллепипеды - если область трехмерна !!!примеры. Каждый из блоков характеризуется координатами в пространстве и размерами, а также информацией о своих границах. Границы блока могут состоять из нескольких частей, каждая из которых представлена !!! Нейман Дирихле , либо является местом соединения с другим блоком.
\par Кроме того, необходима поддержка современного оборудования с неоднородной архитектурой и иерархической организацией памяти. К системам с неоднородной архитектурой относятся, например, вычислительные кластеры, которые используют для расчетов мощности центрального процессора и видеокарт. Иерархическая организация памяти предполагает малые объемы высокоскоростной памяти и большие медленной. Даннный подход подразумевает  экономный подход к использованию доступных ресурсов, по стравнению с разработкой обычных приложений.
\par Программный комплекс должен иметь возможность по окончании вычислений сохранять полученный результат, а также выполнять сохранение текущего состояния в процессе выполнения для последующего возобновления вычислений с любого из сохраненных состояний. Кроме того, необходимо обладать инструментами для запуска расчетов до определенного момента времени с данным шагом или выполнять заданное количество итераций.

\newpage
\section{Компоненты программного комплекса}
\par Приложение состоит из нескольких частей. Каждая из них отвечает за определнную часть работы, либо подготовительной, либо относящейся непосредственно к самим вычислениям, либо реализующей взаимодействие с пользователем.
\par Пользовательский интерфейс позволяет создавать и модифицировать проекты. Кроме того, он дает возможность управлять текущими вычислениями, а именно запускать, прекращать, приостанавливать, а также выполнять сохранение текущего состояния и его загрузку.
\par Часть предварительной обработки выполняет подготовку исходных данных. В ней происходит обработка геометрии и свойств области, производится определение размерности, расположения и размера блоков, граничных условий. Здесь же выполняется распределение блоков по вычислительным устройствам, формируется библиотека пользовательских функций.
\par Парралельный фреймворк и алгоритмы обработки данных составляют основу приложения и занимаются непосредственно самими вычислениями. К алгоритмам обработки данных относятся метод Эйлера, явные схемы Рунге-Кутты, методы Дормана-принца. Стоит отметить, что присутствует возможность расширения списка доступных методов решения.

\newpage
\section{Параллельный фреймворк}
\par Параллельный фреймворк занимается непосредственной реализацией распределенных вычислений. С его помощью выполняется взаимодействие всех вычислительных устройств во время работы приложения. Он же реализует пересылку данных между блоками. Подобные пересылки необходимы в случае, когда одному блоку нужна информация от другого для продолжения собственных вычислений. Кроме того, фреймворк занимается обработкой событий, которые генерирует пользовательский интерфейс, а именно начало вычислений с заданными параметрами, приостановка вычислений, их прикращение, сохранение и загрузка состояний.
\par Крупнозернистый параллелизм осуществляется путем разбиения области на блоки и распредения блоков по вычислительным устойствам еще на этапе подготовки, о чем было изложено ранее. Данный подход позволяет использовать все вычислительные мощности имеющиеся в наличии и равномерно распределять нагрузку на устройства. На одном устройстве может располагаться несколько блоков одновременно, в таком случае их вычисление будет осуществляться в порядке очереди. Немаловажным фактом является то, что в случае если количество блоков значительно (в несколько раз) превышает количество вычислительных устойств, то блоки, которые имеют общие границы разумно располагать на одном устройстве, а блоки не имеющих таких границ - на разных, сохраняя равномерность распределения блоков по устройствам в целом. Такой подход позволяет увеличить скорость расчетов для небольших, по количеству используемых узлов, областей. Достигается это уменьшение времени, которое необходимо на пересылку данных между блоками между итерациями вычислений, так как если блоки расположены на одном вычислительном устройстве, то пересылка не будет выполняться в принципе, потому что данные уже доступны блоку. Очевидно, что в общем случае невозможно распределить все блоки таким образом чтобы пересылок не было вообще, но такой подход позволяет минимизировать количество пересылаемой информации, что положительно сказывается на производительности.


%Для автоматизации моделирования конечно-разностных диффузионных систем необходимо решить несколько задач. Среди них разработка инструмента моделирования уравнений (в том числе и уравнений с запаздыванием), которые были бы распределены в одномерных, двумерхных и трехмерных областях. Каждая область может быть представлена неким набором блоков. Для одномерного случая этими блоками являются отрезки, в случае плоскости - прямоугольник и, если область является трехмерной - параллелепипеды. Каждый из блоков характеризуется своими координатами в пространстве, размерами и информацией о своих границах. Граница блока может быть составлена из нескольких частей !!! Дирихле Нейман граница с другим блоком. !!!РИСУНКИ Кроме того, нужна поддержка соверенного оборудования , а именно неоднородных систем с иерархической организацией памяти. Также требуется высокий уровень абстракции, такой как автоматическая генерация кода по пользовательским функциям. !!! пользовательский интерфейс.

\par 
\end{document}


















