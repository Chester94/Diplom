\documentclass[14pt,a4paper]{extbook}

\usepackage[T2A]{fontenc}
\usepackage[utf8]{inputenc}
\usepackage[russian]{babel}
\usepackage{graphicx}
\usepackage{amsfonts,amssymb,amscd,amsmath}
\usepackage[left=2.5cm, right=2.5cm, top=2.7cm, bottom=2cm]{geometry}

\usepackage{setspace}

\singlespacing

\begin{document}

\begin{center}
{\Large Фреймворк для конечно-разностного моделирования диффузионных задач на
гибридных вычислительных кластерах}
\end{center}


\par Развитие современного общества зачастую ставит перед наукой цели, решение которых требует решения самых разнообразных систем дифференциальных уравнений. В том числе и систем диффузионных уравнений. К таким задачам можно отнести уравнение теплопроводности !!! (примеры других задач). Каждую из них можно решать с помощью численных методов, например, методом Эйлера, многошаговыми методами Рунге-Кутты или методами Дормана-Принца. Подобные вычисления удобно автоматизировать, чтобы в дальнейшем иметь возможность быстро производить расчеты.
\par Для достижения поставленной цели необходимо решить несколько задач. Среди них разработка инструмента моделирования уравнений (в том числе и уравнений с запаздыванием), которые были бы распределены в одномерных, двумерхных и трехмерных областях. Кроме того, нужна поддержка соверенного оборудования с неоднорожной архитектурой.
\end{document}